\documentclass[a4paper,11pt,titlepage]{article}
\usepackage[utf8]{inputenc}
\usepackage{lmodern}
\usepackage[T1]{fontenc}
\usepackage[babel=true]{microtype}
\usepackage[portuguese]{babel}
\usepackage[pdftex]{hyperref}
\usepackage{graphicx}
\usepackage{eurosym}
\usepackage{scrextend}
\usepackage{hyphenat}
\usepackage{url}
\usepackage{hyperref}
\usepackage{float}

\title{\huge \textbf{Protocolo de Ligação de Dados\\[1cm] \Large Relatório intercalar\\[0.7cm]
\includegraphics{res/logo.png}\\[0.7cm] \large Redes de Computadores\\[0.25cm] \small $3^o$ ano\\[0.05cm]Mestrado Integrado em Engenharia Informática e
Computação\\[1cm]}\normalsize Turma 4}

\author{Carolina Moreira\\Daniel Fazeres\\José Peixoto \and 201303494\\201502846\\200603103 \and  up201303494@fe.up.pt\\up201502846@fe.up.pt\\ei12134@fe.up.pt}

\begin{document}
\maketitle

\abstract
(dois parágrafos: um sobre o contexto do trabalho; outro sobre as principais conclusões do relatório)

\section{Introdução}
\iffalse(indicação dos objectivos do trabalho e do relatório; descrição da lógica do relatório com indicações sobre o tipo de informação que poderá ser encontrada em cada uma secções seguintes)\fi

\section{Arquitetura}
\iffalse (blocos funcionais e interfaces)\fi

\section{Estrutura do código}
\iffalse (APIs, principais estruturas de dados, principais funções e sua relação com a arquitetura) \fi

\section{Casos de uso principais}
\iffalse (identificação; sequências de chamada de funções) \fi

\section{Protocolo de ligação lógica}
\iffalse (identificação dos principais aspectos funcionais; descrição da estratégia de implementação destes aspectos com apresentação de extratos de código) \fi

\section{Protocolo de aplicação}
\iffalse (identificação dos principais aspectos funcionais; descrição da estratégia de implementação destes aspectos com apresentação de extractos de código) \fi

\section{Validação}
\iffalse (descrição dos testes efectuados com apresentação quantificada dos resultados, se possível) \fi

\section{Elementos de valorização}
\iffalse (identificação dos elementos de valorização implementados; descrição da estratégia de implementação com apresentação de pequenos extratos de código) \fi

\section{Conclusões}
\iffalse (síntese da informação apresentada nas secções anteriores; reflexão sobre os objectivos de aprendizagem alcançados) \fi


\begin{thebibliography}{9}
\bibitem{lamport93}
  Andrew S. Tanenbaum,
  David J. Wetherall,
  \emph{Computer Networks},
  Prentice Hall, 
  5th edition,
  2011.
\end{thebibliography}

\end{document}

